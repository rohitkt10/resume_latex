% resume.tex
%
% (c) 2002 Matthew Boedicker <mboedick@mboedick.org> (original author) http://mboedick.org
% (c) 2003-2007 David J. Grant <davidgrant-at-gmail.com> http://www.davidgrant.ca
%
% This work is licensed under the Creative Commons Attribution-ShareAlike 3.0 Unported License. To view a copy of this license, visit http://creativecommons.org/licenses/by-sa/3.0/ or send a letter to Creative Commons, 171 Second Street, Suite 300, San Francisco, California, 94105, USA.

\documentclass[letterpaper,11pt]{article}

%-----------------------------------------------------------
%Margin setup

\setlength{\voffset}{0.in}
\setlength{\paperwidth}{8.0in}
\setlength{\paperheight}{12in}
\setlength{\headheight}{0in}
\setlength{\headsep}{0in}
\setlength{\textheight}{7in}
\setlength{\textheight}{11in}
\setlength{\topmargin}{-0.5in}
\setlength{\textwidth}{7.3in}
\setlength{\topskip}{0in}
\setlength{\oddsidemargin}{-0.6in}
\setlength{\evensidemargin}{-0.5in}
%-----------------------------------------------------------
%\usepackage{fullpage}
\usepackage{shading}
\usepackage{hyperref}
\usepackage{color}

%\textheight=9.0in
\pagestyle{empty}
\raggedbottom
\raggedright
\setlength{\tabcolsep}{0in}

%-----------------------------------------------------------
%Custom commands
\newcommand{\resitem}[1]{\item #1 \vspace{-1pt}}
\newcommand{\resheading}[1]{{\large \parashade[.9]{sharpcorners}{\textbf{#1 \vphantom{p\^{E}}}}}}
\newcommand{\ressubheading}[4]{
\begin{tabular*}{6.8in}{l@{\extracolsep{\fill}}r}
		\textbf{#1} & #2 \\
		\textit{#3} & \textit{#4} \\
\end{tabular*}\vspace{-6pt}}
%-----------------------------------------------------------


\begin{document}

%%%%%%%%%%%%%%
%name and address 
%%%%%%%%%%%%%
\begin{tabular*}{7in}{l@{\extracolsep{\fill}}r}
\textbf{\Large Rohit Tripathy}  & +1-765-476-6988\\
\#5 225 South River Road &  rtripath@purdue.edu \\
West Lafayette, IN, USA \\
\end{tabular*}
\\

\vspace{0.05in}

\resheading{Education}
\begin{itemize}
\item
	\ressubheading{Purdue University}{West Lafayette, IN}{PhD., Mechanical Engineering; GPA - 3.8/4.0}{January. 2016 - Dec 2019 (expected)}
	\begin{itemize}
		\resitem{Advisor: Prof. Ilias Bilionis}
		\resitem{Relevant courses: Statistical Machine Learning, Computational Methods in Optimization, Stochastic Processes, Monte Carlo Methods, Decision Theory and Bayesian Statistics, Uncertainty Quantification, Numerical methods.}
	\end{itemize}

\item
	\ressubheading{Purdue University}{West Lafayette, IN}{MS., Mechanical Engineering; GPA - 3.61/4.0}{August 2014-December 2015}
	
\item
	\ressubheading{VIT University}{Vellore, India}{B. Tech., Mechanical Engineering; GPA - 9.04/10.0.}{July 2010-May 2014}

\end{itemize}

\resheading{Work Experience}
\begin{itemize}
\item
	\textit{Givens Associate}, Argonne National Laboratory, Lemont, IL (June 2017 - August 2017).  \\
\item
	\textit{Quantitative Research (Machine Learning) Summer Associate}, JPMorgan Chase \& Co., \\
	New York, NY (May 2018 - \textit{Present})
\end{itemize}

\resheading{Research Experience}
\begin{itemize}
\item
	\ressubheading{Predictive Science Lab, Purdue University}{West Lafayette, IN}{Graduate Research Assistant}{August 2014 - Present}
	\begin{itemize}
		%\resitem{Currently work here under the supervision of Prof. Ilias Bilionis.}
		\resitem{Surrogate  modeling for high-dimensional and multifidelity uncertainty quantification.}
		\resitem{Developed a scalable \textit{Gaussian process regression} technique and currently exploring Deep neural networks for high dimensional input uncertainty quantification tasks.}
	\end{itemize}
\item	
	\ressubheading{Math. and Computer Science (MCS) division, Argonne National Lab}{Lemont, IL}{Givens associate (PhD intern)}{June 2017 - August 2017}
	\begin{itemize}
	\resitem{Explored the use of machine learning methods for wind speed forecasting. In particular, used deep learning techniques for sequence modeling such as LSTMs.}
	\end{itemize}

	
\end{itemize}



%%%%%%%%%%%%%
%%%% publications 
%%%%%%%%%%%%
\resheading{Publications and pre-prints}
\begin{itemize}

\item Rohit Tripathy, Ilias Bilionis, and Marcial Gonzalez. \textit{Gaussian processes with built-in dimensionality reduction: Applications to high-dimensional uncertainty propagation}. Journal of Computational Physics 321 (2016): 191-223.
\item Rohit Tripathy and Ilias Bilionis. \textit{Deep UQ: Learning deep neural network surrogate models for high dimensional uncertainty quantification.} arXiv preprint arXiv:1802.00850 (2018). (under review at \textit{Journal of Computational Physics})
\end{itemize}

\resheading{Recent Talks / Presentations}
\begin{itemize}
\item 
	\ressubheading{SIAM UQ 2018}{Garden Grove, CA}{Learning Deep neural network (DNN) surrogate models for uncertainty quantification.}{April 2018}
\item
	\ressubheading{SIAM AN 2017}{Pittsburgh, PA}{High dimensional multifidelity uncertainty quantification with deep neural networks.}{July 2017}
\item
	\ressubheading{SIAM DR 2017}{Pittsburgh, PA}{Discovering nonlinear active subspaces using deep neural networks.}{July 2017}
\item
	\ressubheading{SIAM CSE 2017}{Atlanta, GA}{Learning multiscale stochastic FEM basis functions with deep neural networks.}{March 2017}
\end{itemize}


%%%%%%%%%%%%%%
%%% School project
%%%%%%%%%%%%%%
\resheading{Other Projects}
\begin{itemize}
\item
	\ressubheading{Uncertainty propagation using kernel ridge regression.}{}{Statistical Machine Learning course, CS 578.}{Jan. 2018 - May 2018}
	\begin{itemize}
		\resitem{Implemented kernel ridge regression and conducted hyperparameter optimization to learn a surrogate for the forward model in a stochastic elliptic partial differential equation. Implemented code in \texttt{Python}.}
	\end{itemize}
\item
	\ressubheading{Optimization over the Stiefel Manifold.}{}{Computational methods in optimization course, CS 520}{Jan. 2016 - May 2016}
	\begin{itemize}
		\resitem{Implemented, in \texttt{Python}, a modified form of gradient descent on manifold space, with update scheme based on the Cayley transform.}
		
	\end{itemize}
\item
	\ressubheading{Finite element solver for a plane stress hypoelasticity problem.}{}{Finite Element Methods course, ME 681.}{Jan. 2015 - May 2015}
	\begin{itemize}
		\resitem{Implemented in \texttt{Python} from scratch a nonlinear finite element solver for 2D hypoelasticity problem for a square plate.}
	\end{itemize}

\item
	\ressubheading{2-D Incompressible Navier Stokes solver}{}{Computational Fluid Dynamics course, ME 614}{Jan. 2015 - May 2015}
	\begin{itemize}
		\resitem{Implemented, in \texttt{Python}, from scratch, a fully conservative finite difference solver with a staggered grid formulation to solve the lid driven cavity problem.}
	\end{itemize}
\end{itemize}




\resheading{Skills}
\begin{description}
\item[Languages (In order of comfort):]
Python, R, MATLAB.

\item[Deep Learning frameworks:]
Theano and tensorflow.

\item[Bayesian Machine Learning frameworks:]
pymc, pymc3 and Edward.
\end{description}

\resheading{Technical Interests}
%\begin{description}
%\item[Academic:] 
Machine Learning, Deep learning and Artificial Intelligence, Data Analysis, Big data, Scalable Inference, Bayesian data analysis, Quantitative finance, Quantitative research, Algorithmic trading. 
%\end{decription}
\resheading{Professional Memberships}
\begin{itemize}
\item Society of Industrial and Applied Mathematics (SIAM) student member [\textit{August 2015- present}].
\item SIAM Purdue chapter Treasurer [\textit{August 2016 - May 2017}].
\end{itemize}

%\resheading{Mentorship Experience}
%Mentored {\color{blue}\underline{\href{https://nanohub.org/groups/ncnsurf}{NCN-SURF}}} student interns in the \href{https://www.predictivesciencelab.org}{Predictive Science Lab} in 2015 and 2016.


\resheading{Links}
\begin{itemize}
	\item \textbf{LinkedIn}: {\color{blue}\underline{\href{http://tinyurl.com/p4myxe8}{http://tinyurl.com/p4myxe8}}}.
	\item \textbf{Speakerdeck}: {\color{blue}\underline{\href{http://speakerdeck.com/rohitkt10}{http://speakerdeck.com/rohitkt10}}}.
	\item \textbf{Active subspace project github}: {\color{blue}\underline{\href{https://github.com/PredictiveScienceLab/py-aspgp}{https://github.com/PredictiveScienceLab/py-aspgp}}}.
\end{itemize}
\end{document}